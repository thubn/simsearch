% -----------------------------
% Basic document setup (load first)
% -----------------------------
\usepackage[a4paper]{geometry}  % Page layout settings
\usepackage[german,english]{babel}  % Language support
\usepackage{csquotes}  % Required by babel/biblatex

% -----------------------------
% Font and encoding packages
% -----------------------------
\usepackage{fontspec}  % XeLaTeX font support
\usepackage{textcomp}  % Additional symbols
\usepackage{gensymb}  % General symbols
%\usepackage{FiraSans}
%\usepackage{tgpagella}
%\usepackage{tgbonum}
%\usepackage{libertine}

% -----------------------------
% Math packages
% -----------------------------
\usepackage{amsmath}
\usepackage{amssymb}
%\usepackage{amsthm}
\usepackage{amsfonts}
\usepackage{esvect}
\usepackage{commath}
\usepackage{tcolorbox}

% -----------------------------
% Graphics and color
% -----------------------------
\usepackage{graphicx}  % Image support
\usepackage{xcolor}    % Color support (includes color package functionality)
\usepackage{subfigure}
\usepackage{tikz}
\usepackage{pgf-umlcd}
\usepackage{float}
\usepackage{pdfpages}

% -----------------------------
% Plots
% -----------------------------

\usepackage{pgfplots}
\usepackage{pgfplotstable}
\pgfplotsset{compat=newest}

% -----------------------------
% Lists and algorithms
% -----------------------------
\usepackage{enumerate}
%\usepackage[plain,chapter]{algorithm}
\usepackage{algorithmic}
\usepackage{listings}
\usepackage{booktabs}

% -----------------------------
% Theorem environments
% -----------------------------
\usepackage[amsmath,thmmarks]{ntheorem}

% -----------------------------
% Typography improvements
% -----------------------------
\usepackage{microtype}  % Better typography
\usepackage{sectsty}    % Section title formatting

% -----------------------------
% Formatting
% -----------------------------
\usepackage{seqsplit}

% -----------------------------
% Captions and URLs
% -----------------------------
\usepackage[margin=0pt,font=small,labelfont=bf]{caption}
\usepackage{url}

% -----------------------------
% Bibliography setup (load late)
% -----------------------------
\usepackage[backend=biber,
    style=numeric,      % numbered citations
    sorting=none,       % numbers appear in citation order
    urldate=long,       % includes date when website was accessed
    hyperref=true]{biblatex}
\addbibresource{literatur/diplom.bib}

% -----------------------------
% Load hyperref last
% -----------------------------

\usepackage[hidelinks]{hyperref}

% -----------------------------
% Package-specific settings
% -----------------------------

% set default fonts
\setmainfont{Linux Libertine O}

% dont write CHAPTER at the beginning of every chapter. Just the number and chapter title.
\makeatletter
\def\@makechapterhead#1{%
    \vspace*{50\p@}%
    {\parindent \z@ \raggedright
        \ifnum \c@secnumdepth >\m@ne
            \huge\bfseries \thechapter\space\space%
        \fi
        \interlinepenalty\@M
        \huge \bfseries #1\par\nobreak
        \vskip 40\p@
    }}
\makeatother

% Change sizes for different heading levels
%\chapterfont{\huge}          % For chapters
\sectionfont{\Large}         % For sections
\subsectionfont{\large}      % For subsections
\subsubsectionfont{\normalsize}  % For subsubsections

\lstset{
    frame=none,
    breaklines=true,
    basicstyle=\ttfamily\footnotesize,
    keywordstyle=\color{blue},
    commentstyle=\color{green!60!black},
    stringstyle=\color{red},
    numbers=left,
    numberstyle=\tiny
    %numbersep=5pt
}

\newlength{\widefigwidth}
\setlength{\widefigwidth}{\paperwidth}
\addtolength{\widefigwidth}{-4cm}

% -----------------------------
% Set theorem styles
% -----------------------------
% Define the theorem styles
% \theoremstyle{definition}  % Style for examples (upright font)
% \newtheorem{example}{Example}[section]

% \theoremstyle{remark}     % Style for remarks (upright font, no special formatting)
% \newtheorem{remark}{Remark}[section]

% Define the theorem styles
\tcbuselibrary{theorems}
\tcbuselibrary{skins}  % For enhanced style
\tcbuselibrary{breakable}  % For allowing page breaks
\tcbuselibrary{listings}  % For code formatting
\newtcbtheorem[number within=chapter]{example}{Example}{
    enhanced,
    colback=white,
    colframe=black!50,
    fonttitle=\bfseries,
    colbacktitle=white,
    coltitle=black,
    attach boxed title to top left={yshift=-2mm, xshift=2mm},
    boxed title style={
            size=small,
            colframe=white,
            colback=white
        },
    before upper={\parindent15pt}
}{ex}

\newtcbtheorem[number within=chapter]{remark}{Remark}{
    enhanced,
    colback=gray!5,
    colframe=gray!50,
    fonttitle=\bfseries,
    colbacktitle=gray!5,
    coltitle=black,
    attach boxed title to top left={yshift=-2mm, xshift=2mm},
    boxed title style={
            size=small,
            colframe=gray!5,
            colback=gray!5
        },
    before upper={\parindent15pt}
}{rem}

% Define the algorithm style with listing support
\newtcbtheorem[number within=section]{algorithm}{Algorithm}{
  enhanced,
  colback=white,
  colframe=black!50,
  fonttitle=\bfseries,
  colbacktitle=white,
  coltitle=black,
  left=2mm,          % Added for internal margin control
  right=2mm,         % Added for internal margin control
  top=0mm,           % Added for internal margin control
  bottom=0mm,        % Added for internal margin control
  before skip=3pt,   % Added for external margin control
  after skip=3pt,    % Added for external margin control
  attach boxed title to top left={yshift=-2mm, xshift=2mm},
  boxed title style={
    size=small,
    colframe=white,
    colback=white
  },
  listing engine=listings,  % Specify listing engine
  listing only,            % Important!
  listing options={
    language=C++,       % Specify the language
    numbers=left,
    numberstyle=\tiny,
    basicstyle=\ttfamily\small,
    keywordstyle=\color{blue}\bfseries,
    commentstyle=\color{green!60!black},
    showstringspaces=false,
    breaklines=true,
    postbreak=\mbox{\textcolor{red}{$\hookrightarrow$}\space},
  },
  breakable=false,
  before upper={\parindent15pt}
}{alg}

\makeatletter
\newcommand\tcb@cnt@algorithmautorefname{Algorithm}
\makeatother
% change headings to sans serif
% \chapterfont{\selectfont}
% \allsectionsfont{\sffamily}

% % Theorem-Optionen %
% \theoremseparator{.}
% \theoremstyle{change}
% \newtheorem{theorem}{Theorem}[section]
% \newtheorem{satz}[theorem]{Satz}
% \newtheorem{lemma}[theorem]{Lemma}
% \newtheorem{korollar}[theorem]{Korollar}
% \newtheorem{proposition}[theorem]{Proposition}
% % Ohne Numerierung
% \theoremstyle{nonumberplain}
% \renewtheorem{theorem*}{Theorem}
% \renewtheorem{satz*}{Satz}
% \renewtheorem{lemma*}{Lemma}
% \renewtheorem{korollar*}{Korollar}
% \renewtheorem{proposition*}{Proposition}
% % Definitionen mit \upshape
% \theorembodyfont{\upshape}
% \theoremstyle{change}
% \newtheorem{definition}[theorem]{Definition}
% \theoremstyle{nonumberplain}
% \renewtheorem{definition*}{Definition}
% % Kursive Schrift
% \theoremheaderfont{\itshape}
% \newtheorem{notation}{Notation}
% \newtheorem{konvention}{Konvention}
% \newtheorem{bezeichnung}{Bezeichnung}
% \theoremsymbol{\ensuremath{\Box}}
% \newtheorem{beweis}{Beweis}
% \theoremsymbol{}
% \theoremstyle{change}
% \theoremheaderfont{\bfseries}
% \newtheorem{bemerkung}[theorem]{Bemerkung}
% \newtheorem{beobachtung}[theorem]{Beobachtung}
% \newtheorem{beispiel}[theorem]{Beispiel}
% \newtheorem{problem}{Problem}
% \theoremstyle{nonumberplain}
% \renewtheorem{bemerkung*}{Bemerkung}
% \renewtheorem{beispiel*}{Beispiel}
% \renewtheorem{problem*}{Problem}

% % Algorithmen anpassen %
% \renewcommand{\algorithmicrequire}{\textit{Eingabe:}}
% \renewcommand{\algorithmicensure}{\textit{Ausgabe:}}
% \floatname{algorithm}{Algorithmus}
% \renewcommand{\listalgorithmname}{Algorithmenverzeichnis}
% \renewcommand{\algorithmiccomment}[1]{\color{grau}{// #1}}

% % Zeilenabstand einstellen %
% \renewcommand{\baselinestretch}{1.25}
% % Floating-Umgebungen anpassen %
% \renewcommand{\topfraction}{0.9}
% \renewcommand{\bottomfraction}{0.8}
% % Abkuerzungen richtig formatieren %
% \usepackage{xspace}
% \newcommand{\vgl}{vgl.\@\xspace}
% \newcommand{\zB}{z.\nolinebreak[4]\hspace{0.125em}\nolinebreak[4]B.\@\xspace}
% \newcommand{\bzw}{bzw.\@\xspace}
% \newcommand{\dahe}{d.\nolinebreak[4]\hspace{0.125em}h.\nolinebreak[4]\@\xspace}
% \newcommand{\etc}{etc.\@\xspace}
% \newcommand{\evtl}{evtl.\@\xspace}
% \newcommand{\ggf}{ggf.\@\xspace}
% \newcommand{\bzgl}{bzgl.\@\xspace}
% \newcommand{\so}{s.\nolinebreak[4]\hspace{0.125em}\nolinebreak[4]o.\@\xspace}
% \newcommand{\iA}{i.\nolinebreak[4]\hspace{0.125em}\nolinebreak[4]A.\@\xspace}
% \newcommand{\sa}{s.\nolinebreak[4]\hspace{0.125em}\nolinebreak[4]a.\@\xspace}
% \newcommand{\su}{s.\nolinebreak[4]\hspace{0.125em}\nolinebreak[4]u.\@\xspace}
% \newcommand{\ua}{u.\nolinebreak[4]\hspace{0.125em}\nolinebreak[4]a.\@\xspace}
% \newcommand{\og}{o.\nolinebreak[4]\hspace{0.125em}\nolinebreak[4]g.\@\xspace}
% \newcommand{\oBdA}{o.\nolinebreak[4]\hspace{0.125em}\nolinebreak[4]B.\nolinebreak[4]\hspace{0.125em}d.\nolinebreak[4]\hspace{0.125em}A.\@\xspace}
% \newcommand{\OBdA}{O.\nolinebreak[4]\hspace{0.125em}\nolinebreak[4]B.\nolinebreak[4]\hspace{0.125em}d.\nolinebreak[4]\hspace{0.125em}A.\@\xspace}

% Leere Seite ohne Seitennummer, naechste Seite rechts
\newcommand{\blankpage}{
    \clearpage{\pagestyle{empty}\cleardoublepage}
}

% % Keine einzelnen Zeilen beim Anfang eines Abschnitts (Schusterjungen)
% \clubpenalty = 10000
% % Keine einzelnen Zeilen am Ende eines Abschnitts (Hurenkinder)
% \widowpenalty = 10000 \displaywidowpenalty = 10000

% EOF
